\documentclass[journal,12pt,onecolumn]{IEEEtran}

\usepackage{setspace}
\usepackage{gensymb}
\singlespacing


\usepackage[cmex10]{amsmath}

\usepackage{amsthm}

\usepackage{mathrsfs}
\usepackage{txfonts}
\usepackage{stfloats}
\usepackage{bm}
\usepackage{cite}
\usepackage{cases}
\usepackage{subfig}

\usepackage{longtable}
\usepackage{multirow}

\usepackage{enumitem}
\usepackage{mathtools}
\usepackage{steinmetz}
\usepackage{tikz}
\usepackage{circuitikz}
\usepackage{verbatim}
\usepackage{tfrupee}
\usepackage[breaklinks=true]{hyperref}
\usepackage{graphicx}
\usepackage{tkz-euclide}

\usetikzlibrary{calc,math}
\usepackage{listings}
    \usepackage{color}                                            %%
    \usepackage{array}                                            %%
    \usepackage{longtable}                                        %%
    \usepackage{calc}                                             %%
    \usepackage{multirow}                                         %%
    \usepackage{hhline}                                           %%
    \usepackage{ifthen}                                           %%
    \usepackage{lscape}     
\usepackage{multicol}
\usepackage{chngcntr}

\DeclareMathOperator*{\Res}{Res}

\renewcommand\thesection{\arabic{section}}
\renewcommand\thesubsection{\thesection.\arabic{subsection}}
\renewcommand\thesubsubsection{\thesubsection.\arabic{subsubsection}}

\renewcommand\thesectiondis{\arabic{section}}
\renewcommand\thesubsectiondis{\thesectiondis.\arabic{subsection}}
\renewcommand\thesubsubsectiondis{\thesubsectiondis.\arabic{subsubsection}}


\hyphenation{op-tical net-works semi-conduc-tor}
\def\inputGnumericTable{}                                 %%

\lstset{
%language=C,
frame=single, 
breaklines=true,
columns=fullflexible
}
\begin{document}


\newtheorem{theorem}{Theorem}[section]
\newtheorem{problem}{Problem}
\newtheorem{proposition}{Proposition}[section]
\newtheorem{lemma}{Lemma}[section]
\newtheorem{corollary}[theorem]{Corollary}
\newtheorem{example}{Example}[section]
\newtheorem{definition}[problem]{Definition}

\newcommand{\BEQA}{\begin{eqnarray}}
\newcommand{\EEQA}{\end{eqnarray}}
\newcommand{\define}{\stackrel{\triangle}{=}}
\bibliographystyle{IEEEtran}
\providecommand{\mbf}{\mathbf}
\providecommand{\pr}[1]{\ensuremath{\Pr\left(#1\right)}}
\providecommand{\qfunc}[1]{\ensuremath{Q\left(#1\right)}}
\providecommand{\sbrak}[1]{\ensuremath{{}\left[#1\right]}}
\providecommand{\lsbrak}[1]{\ensuremath{{}\left[#1\right.}}
\providecommand{\rsbrak}[1]{\ensuremath{{}\left.#1\right]}}
\providecommand{\brak}[1]{\ensuremath{\left(#1\right)}}
\providecommand{\lbrak}[1]{\ensuremath{\left(#1\right.}}
\providecommand{\rbrak}[1]{\ensuremath{\left.#1\right)}}
\providecommand{\cbrak}[1]{\ensuremath{\left\{#1\right\}}}
\providecommand{\lcbrak}[1]{\ensuremath{\left\{#1\right.}}
\providecommand{\rcbrak}[1]{\ensuremath{\left.#1\right\}}}
\theoremstyle{remark}
\newtheorem{rem}{Remark}
\newcommand{\sgn}{\mathop{\mathrm{sgn}}}
\providecommand{\abs}[1]{\ensuremath{\left\vert#1\right\vert}}
\providecommand{\res}[1]{\Res\displaylimits_{#1}} 
\providecommand{\norm}[1]{\ensuremath{\left\lVert#1\right\rVert}}
%\providecommand{\norm}[1]{\lVert#1\rVert}
\providecommand{\mtx}[1]{\mathbf{#1}}
\providecommand{\mean}[1]{E\ensuremath{\left[ #1 \right]}}
\providecommand{\fourier}{\overset{\mathcal{F}}{ \rightleftharpoons}}
%\providecommand{\hilbert}{\overset{\mathcal{H}}{ \rightleftharpoons}}
\providecommand{\system}{\overset{\mathcal{H}}{ \longleftrightarrow}}
	%\newcommand{\solution}[2]{\textbf{Solution:}{#1}}
\newcommand{\solution}{\noindent \textbf{Solution: }}
\newcommand{\cosec}{\,\text{cosec}\,}
\newcommand{\R}{\mathbb{R}}
\providecommand{\dec}[2]{\ensuremath{\overset{#1}{\underset{#2}{\gtrless}}}}
\newcommand{\myvec}[1]{\ensuremath{\begin{pmatrix}#1\end{pmatrix}}}
\newcommand{\mydet}[1]{\ensuremath{\begin{vmatrix}#1\end{vmatrix}}}
\numberwithin{equation}{subsection}
\makeatletter
\@addtoreset{figure}{problem}
\makeatother
\let\StandardTheFigure\thefigure
\let\vec\mathbf
\renewcommand{\thefigure}{\theproblem}
\def\putbox#1#2#3{\makebox[0in][l]{\makebox[#1][l]{}\raisebox{\baselineskip}[0in][0in]{\raisebox{#2}[0in][0in]{#3}}}}
     \def\rightbox#1{\makebox[0in][r]{#1}}
     \def\centbox#1{\makebox[0in]{#1}}
     \def\topbox#1{\raisebox{-\baselineskip}[0in][0in]{#1}}
     \def\midbox#1{\raisebox{-0.5\baselineskip}[0in][0in]{#1}}
\vspace{3cm}
\onecolumn
\title{EE5609: Matrix Theory\\
          Assignment 12\\}
\author{Lt Cdr Atul Mahajan\\MTech Artificial Intelligence\\AI20MTECH13001 }
\maketitle
\bigskip
\renewcommand{\thefigure}{\theenumi}
\renewcommand{\thetable}{\theenumi}
Download codes from 
%
\begin{lstlisting}
https://github.com/Atul 191/EE5609/Assignment12
\end{lstlisting}
%
 
\section{Question}
Let A,B be $n \times n$ matrices such that $BA+ B^2= I-BA^2$ where I is the $n\times n$ identity matrix. Which of the following is always correct
\begin{enumerate}
    \item A is non singular
    \item B is non singular
    \item A+B is non singular
    \item AB is non singular
\end{enumerate}
%
\section{Solution}
\begin{longtable}{|p{5cm}|p{13cm}|}
\hline
\textbf{Statement} &\textbf{Solution}\\
\hline
Given Condition&
\parbox{12cm}{\begin{align}
BA+ B^2= I-BA^2 \label{eq1}
\end{align}}\\
\hline
Solution by Theory&
\parbox{12cm}{We will first provide theoretical proof }\\
\hline
Theory&
\parbox{12cm}{\text{As per definition of invertible matrix,A matrix 'B' in our case is defined}\\
 \text{as invertible if there exists left and right inverse of B such that BC=CB=I } \\
\text{In that case C is called the two sided inverse of B and B is said to be }\\
\text{invertible.}\\
\text{Now refer\eqref{eq1} we get}
\begin{align}
 BA+ B^2= I-BA^2\\
 \implies BA+B^2+BA^2=I\\
 \implies I=B\brak{A+B+A^2}\label{eqB}\\
 \end{align}
 \text{Let C= \brak{A+B+A^2} rewrite \eqref{eqB} as}
  \begin{align}
  I=BC\label{eqC}
  \end{align}
  \text{Also}\\
 \begin{align}
 I=\brak{A+B+A^2}B\label{eqD}
 \end{align}
 \text{Let D= \brak{A+B+A^2} rewrite \eqref{eqD} as}\\
 \begin{align}
 I=DB\label{eqE}
 \end{align}
 \text{Now we can write}
 \begin{align}
    D= DI
   \end{align}
  \text{ Ref \eqref{eqC}}
  \begin{align}
     =D\brak{BC}\\
    =\brak{DB}C\\
  \end{align}
  \text{ Ref \eqref{eqE}}
  \begin{align}
      =IC\\
      =C\\
      \implies D=C
  \end{align}
  \text{Hence by definition stated above we imply that }\\
  \text{ Left inverse=Right inverse.}\\
  \text{So by looking at \eqref{eqB},we imply that B has a left and right inverse}\\
 \begin{align}
 \implies I=BB^{-1}\\
 \implies \text{B is invertible}
 \end{align}
 \text{$\therefore$ B is non singular.\\Hence Option 2 is correct}
}
\\
\hline
Solution by examples&
\parbox{12cm}{We will check each respective options through examples}\\
\hline
Option 3&
\parbox{12cm}{\text{Let us take}
\begin{align}
A=\myvec{1&0\\0&1}\\
B=\myvec{-1&0\\0&-1}
\end{align}
\text{ Take L.H.S of \eqref{eq1}}
\begin{align}
\myvec{-1&0\\0&-1}\myvec{1&0\\0&1}+\myvec{-1&0\\0&-1}\myvec{-1&0\\0&-1}\\
=\myvec{0&0\\0&0}\label{eq2}
\end{align}
\text{ Take R.H.S of \eqref{eq1}}
\begin{align}
 \myvec{1&0\\0&1}-\myvec{-1&0\\0&-1}\myvec{1&0\\0&1}\myvec{1&0\\0&1}\\
 =\myvec{0&0\\0&0}\label{eq3}
\end{align}
\text{ Our assumption satisfies \eqref{eq1}.}
\text{Now}
\begin{align}
    A+B=  \myvec{1&0\\0&1} + \myvec{-1&0\\0&-1}\\
    = \myvec{0&0\\0&0}
\end{align}
\text{$\therefore \mydet{A+B}=0$  the respective option is Singular. Hence Option 3 is incorrect}
}
\\
\hline
Option 1&
\parbox{12cm}{\text{ Now let us take} 
\begin{align}
 A=\myvec{0&0\\0&0}
 B=\myvec{-1&0\\0&-1} \label{eq4}
\end{align}
\text{Substituting\eqref{eq4} in \eqref{eq1}}\\
\text{Take L.H.S of \eqref{eq1}}
\begin{align}
\myvec{-1&0\\0&-1}\myvec{0&0\\0&0}+\myvec{-1&0\\0&-1}\myvec{-1&0\\0&-1}\\
 = \myvec{1&0\\0&1}
\end{align}
\text{Take R.H.S of \eqref{eq1}}
\begin{align}
 \myvec{1&0\\0&1}-\myvec{-1&0\\0&-1}\myvec{0&0\\0&0}\myvec{0&0\\0&0}\\
 =\myvec{1&0\\0&1}\label{eq3}
\end{align}
\text{Our assumption satisfies \eqref{eq1}}\\
\text{But $\mydet{A}=0$}\\
\tect{$\therefore$  the respective option is Singular. Hence Option 1 is incorrect}
}
\\
\hline
 Option 4&
\parbox{12cm}{\text{Similarly} 
\begin{align}
AB= \myvec{0&0\\0&0}\myvec{-1&0\\0&-1}\\
 = \myvec{0&0\\0&0}
\end{align}
\text{Here also $\mydet{AB}=0$}\\
\text{$\therefore$  the AB option is also Singular. Hence Option 4 is incorrect also}
}
\\
\hline
Correct Answer&
\parbox{12cm}{\text{So we conclude that Option 2 is correct by eliminating other options}}\\
\hline
\caption*{Table1:Solution}
\end{longtable}
\end{document}